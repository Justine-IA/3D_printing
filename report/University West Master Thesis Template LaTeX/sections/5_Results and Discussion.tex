% CREATED BY C. DITTMER, 2023

\clearpage

\fancyhead{} % clear all header fields
\fancyhead[C]{\footnotesize\textit{Degree Project for Master of Science with specialization in Robotics}\\ \footnotesize\textbf{\myTitle - Results and Discussion}}

\chapter{Results and Discussion}\thispagestyle{fancy}
%\titleformat{\chapter}[hang]{\Huge\bfseries}{\thechapter\hsp\textcolor{black}\hsp}{10pt}{\Huge\bfseries}
The Results and discussion section presents a detailed and objective description of the results. This can be done by showing tables, figures which are numbered and giv-en explanatory figure/table texts. Figure numbering is below the figure while the tables get their numbering above.  If tables or figures are too big or too many, the most interesting are added in this section and the rest must be placed in the appendix, see Appendix. In graphs don’t forget to state units and to use the figure area optimal. Avoid strange/hard reading colours and symbols; it must be easy for the reader to understand the graphs. This section must also conclude the results and findings (or if too extensive in a new section for result discussion). The purpose is to discuss and analyse the meaning of the results related to the theoretical framework, related jobs and methodology. Questions to be asked when you write this section: what do the readings say, how likely are they, can they be generalized, how significant are the re-sults, outliers, etc. In this part, personal reflections are allowed.

\begin{comment}
To write notes or comments that isn't published in the pdf.
\end{comment}
