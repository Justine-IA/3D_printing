% CREATED BY C. DITTMER, 2023

\clearpage

\fancyhead{} % clear all header fields
\fancyhead[C]{\footnotesize\textit{Degree Project for Master of Science with specialization in Robotics}\\ \footnotesize\textbf{\myTitle - Introduction}}
\fancyfoot[C]{\thepage}

\chapter{Introduction}\thispagestyle{fancy}

\section{Introduction}

\subsection{Too deep structure}
The introduction is a section that must exist. The purpose is to create an interest and understanding of the problem. The introduction should introduce the project, study, or whatever the report is meant to describe. Begin by explaining the area or put the research question into context. The introduction should introduce the reader to the area and logically justify the problem that the report treats. The introduction starts broad and describes the area in general and then concludes the specific problem area. The problem area description in a degree project shall be particularly well described, justified and explained. It must be easy for other engineers with a similar background to read and understand the introduction without being an expert within the filed. The beginning can be written as a single section, but can also be broken down into sub-sections with sub-headings (e.g. 1.1, 1.2, etc.). A sub-section must not be less than four lines. Whatever structure you choose, it must be easy for the reader to follow and understand your thoughts. An introduction of less than one page is usually not ok for a degree project. 

\begin{comment}
To write notes or comments that isn't published in the pdf.
\end{comment}

\section{Aim}
To clarify the presentation of the problem the writer must state some aims or objectives of the study. Make sure that the aims are clear, measurable and not too many or spread out as different aims all over the report.  Be aware that all aims should be an-swered in the report. Usually, there exist aims at different levels. Try to formulate the overall aim as the aim of the thesis. Intermediate targets to reach the overall aim can be described in the method (if appropriate) or later in the report. A subsection of the introduction called aim must exist in the final report. Also, note that the aims of the project are stated in the project description. However, they might have changed dur-ing the project and the finalised aims shall be considered in the report. 

\begin{comment}
To write notes or comments that isn't published in the pdf.
\end{comment}

\section{Language}
Use the built-in English British dictionary when you write. For most of us, including supervisors and examiners, English is a second language which indeed will generate a non-perfect English report. However, do not distribute reports with easy to find er-rors. Use the built-in mechanisms to check for spelling and grammatical errors and follow the template for formatting. The task for the supervisor or examiner is not to correct misspellings, formats, bad figures, lack of figure texts, non-consistent refer-ence lists etc. At master level, this is considered as fundamental knowledge and should be handled without any problem by the student. If such simple mistakes are found in the text it will be immediately returned to the author without any further comments. 

\begin{comment}
To write notes or comments that isn't published in the pdf.
\end{comment}

\section{Program code}
A project might aim at developing software’s for robots, machine visions, PLC etc. It is then not unusual that the author wishes to discuss the contents of the code. How-ever, avoid detailed code descriptions as far as possible in a report. The reader might be unfamiliar with the specific language/syntax used. Further, it might be inconven-ient to read a larger section of code in a report. \\ 
It is better to write in more general terms such as pseudo code or to show a flow dia-gram of the solution. However, if it is necessary to show program code it must un-derstandable for the reader. If it’s too extensive, more than half a page, it is more suitable for an appendix. Note that extensive program listings are usually not ok even for an appendix. Extensive code sections are usually best distributed as files and can-not be part of the report. For program listing use style “ThesisCode” the numbering of lines will help you to refer to a specific line in the code. Note the … on line 2 in Figure 2 the purpose is to show only the necessary part of the code and the rest is removed. A program listing shall have a figure text as a normal figure. 

\begin{comment}
To write notes or comments that isn't published in the pdf.
\end{comment}

\section{Cross-references}
A reference to a heading, picture, table or bibliography is called a cross-reference in Word. To be able to handle your thesis report or other extensive material in a profes-sional way cross-references must be used. A cross-reference is a symbolic link to a source, e.g., a heading. When you change the text in a source the cross-reference text will be updated automatically. However, the cross-reference text might not be updat-ed directly you might have to reopen the document to see the new result. To force an update, it is possible to right-click a cross-reference and select update. To update all cross-references in the entire document, select all text by the shortcut [CTRL]+[A]. When all text is selected press [F9], in the following dialog box select “update entire table” to update everything. In this way, the table of context page is also updated with new/removed sections and with correct page numbers.\\
If somethings go wrong with a cross-reference you will see something like this: Error! Reference source not found. This could happen if the source is deleted, you must then delete the obsolete cross-reference and insert a new. This could also hap-pen due to other reasons not explained here but the first thing to do is to update the entire document save it and reopen it. If the problem remains then delete the cross-reference and insert a new one. 

\begin{comment}
To write notes or comments that isn't published in the pdf.
\end{comment}
